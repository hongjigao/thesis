\chapter{Preliminaries and Related Work}
\label{sec:prelim}
\subsection{Hierarchical matrices}
\label{subsec:h-matrix}

Hierarchical matrices ($\mathcal{H}$-matrices) are data-sparse representations
for large, structured matrices arising in applications such as elliptic partial
differential equations, boundary integral equations, and kernel methods
\cite{HackbuschBorm2002,Bebendorf2008,Hackbusch2015}. The main idea is to
exploit the fact that matrix blocks corresponding to well-separated index
clusters are often numerically low rank, while blocks corresponding to
near-field interactions are not.

More precisely, one first organizes the index set
$\mathcal{I} = \{1,\dots,N\}$ into a cluster tree based on geometric or
algebraic information, and then constructs a block cluster tree that partitions
the $N\times N$ matrix into blocks. An admissibility condition---typically
expressing that the underlying index clusters are well separated---is used to
classify blocks as \emph{far-field} or \emph{near-field}. Near-field blocks are
stored explicitly as dense matrices, while far-field blocks are compressed by
low-rank approximations, often obtained by SVD, RRQR, or SRRQR.

Under mild assumptions on the underlying problem and the admissibility
condition, $\mathcal{H}$-matrices achieve almost linear storage and arithmetic
complexity. For example, for a second-order elliptic operator in three
dimensions, one can typically store the corresponding stiffness matrix in
$\mathcal{O}(N \log N)$ memory and apply it to a vector in
$\mathcal{O}(N \log^2 N)$ time, where $N$ is the number of degrees of freedom
\cite{Hackbusch2015}. In the language of Section~\ref{subsec:asymptotic}, this
represents a substantial improvement over the $\Theta(N^2)$ storage and
$\Theta(N^2)$ matrix-vector multiplication cost of a dense matrix.

The hierarchical organization is particularly flexible: blocks can be further
subdivided, admissibility can be adapted to the problem at hand, and different
low-rank approximation techniques can be used in different parts of the
matrix. This flexibility makes $\mathcal{H}$-matrices a natural tool for
compressing large matrices and tensors in many-body problems.

\subsection{$\mathcal{H}^2$-matrices}
\label{subsec:h2-matrix}

While $\mathcal{H}$-matrices already reduce storage and computation costs
significantly, they store separate low-rank bases for different far-field
blocks, which leads to redundancy. $\mathcal{H}^2$-matrices refine this idea by
introducing \emph{nested} cluster bases that are shared across blocks
\cite{BormGrasedyckHackbusch2003,Hackbusch2015}. In an $\mathcal{H}^2$-matrix,
each cluster in the index tree is associated with a basis that spans the
relevant far-field interactions, and bases at coarse levels are represented in
terms of bases at finer levels via small transfer matrices.

This nested structure eliminates much of the redundancy present in standard
$\mathcal{H}$-matrices. Under suitable assumptions on the kernel or operator,
$\mathcal{H}^2$-matrices can achieve linear or nearly linear complexity:
\begin{equation}
  \mathcal{O}(N) \quad\text{for storage}, \qquad
  \mathcal{O}(N) \quad\text{for matrix-vector multiplication},
\end{equation}
again up to logarithmic factors and assuming bounded ranks. From the asymptotic
viewpoint, this represents an improvement from $\mathcal{O}(N \log N)$ or
$\mathcal{O}(N \log^2 N)$ to (essentially) $\mathcal{O}(N)$, while maintaining
a prescribed accuracy in the relevant matrix norm.

$\mathcal{H}^2$-matrices are particularly effective for discretizations of
translation-invariant or asymptotically smooth kernels, such as Green's
functions of elliptic operators and Coulomb-like interactions. In later
chapters we will make use of hierarchical and $\mathcal{H}^2$-matrix ideas to
compress large matrices and tensors arising in electronic-structure theory,
and we will analyze their storage, computational complexity, and approximation
errors using the asymptotic framework introduced in
Section~\ref{sec:prelim-convergence}.

\section{Quantum Chemistry Background}
\label{sec:prelim-chem}

We now review the quantum chemistry background relevant to this work, focusing
on the tensor structures and electronic-structure methods that will be used in
later sections. We first introduce the full configuration interaction (FCI)
tensor and the electron repulsion integral (ERI) tensor, and then briefly
review the Hartree--Fock (HF) method and second-order M{\o}ller--Plesset
perturbation theory (MP2). Throughout, we emphasize the asymptotic behavior of
the computational cost in terms of the number of basis functions and electrons.

\subsection{Full configuration interaction tensor}
\label{subsec:fci-tensor}

Consider an $N$-electron system in a finite one-particle basis of $M$
spin-orbitals $\{\chi_p\}_{p=1}^M$. Within this basis, the electronic
wave function can be expanded in a basis of Slater determinants
$\{\Phi_I\}$ constructed by occupying $N$ spin-orbitals out of $M$:
\begin{equation}
  \Psi
  = \sum_{I} C_I \Phi_I,
  \qquad
  \Phi_I = \frac{1}{\sqrt{N!}}
  \det\bigl[\chi_{i_1}(1)\,\chi_{i_2}(2)\,\dots\,\chi_{i_N}(N)\bigr],
\end{equation}
where $I = (i_1,\dots,i_N)$ indexes the occupied spin-orbitals in determinant
$\Phi_I$ and $C_I$ are configuration interaction (CI) coefficients. In full
configuration interaction (FCI), the sum runs over all determinants consistent
with the specified number of electrons and spin symmetry, leading to a formally
exact solution of the nonrelativistic electronic Schr\"odinger equation within
the chosen basis \cite{SzaboOstlund1989,HelgakerJorgensenOlsen2000}.

The dimension of the determinant basis is
\begin{equation}
  N_\mathrm{det}
  = \binom{M}{N_\alpha} \binom{M}{N_\beta},
\end{equation}
where $N_\alpha$ and $N_\beta$ are the numbers of $\alpha$- and $\beta$-spin
electrons, respectively. As $M$ grows, $N_\mathrm{det}$ increases
combinatorially, which leads to an exponential scaling of the FCI cost with
respect to system size. In the asymptotic language of
Section~\ref{subsec:asymptotic}, the storage required for the CI coefficients
is $\Theta(N_\mathrm{det})$, which behaves roughly like
$\exp(\mathcal{O}(M))$ for fixed electron fraction.

It is often convenient to view the collection of CI coefficients as a tensor,
the \emph{FCI tensor}. For example, in an occupation-number representation the
FCI tensor has one index per spin-orbital,
\begin{equation}
  C_{n_1 n_2 \dots n_M},
  \qquad n_p \in \{0,1\},
\end{equation}
subject to the constraint $\sum_p n_p = N$. Alternatively, one can group
spin-orbitals into $\alpha$- and $\beta$-strings and fold the CI vector into a
matrix with indices $(I_\alpha, I_\beta)$, where each index labels an
occupation pattern of $\alpha$- or $\beta$-spin orbitals. Regardless of the
specific indexing, the resulting FCI tensor is extremely high-dimensional and
dense, and serves as a prototypical example of an object with exponential
storage requirements.

Because of this unfavorable asymptotic behavior, practical electronic-structure
calculations rely on truncated CI expansions (CISD, CISDTQ, etc.), coupled-cluster
theory, selected CI, density-matrix renormalization group (DMRG), tensor
network states, and various low-rank or sparsity-exploiting wave function
approximations; see, e.g., \cite{SzaboOstlund1989,HelgakerJorgensenOlsen2000,
SherrillSchaefer1999,ChanHeadGordon2002,Schollwock2011} for reviews. In later
sections we will revisit the FCI tensor as a motivating example for hierarchical
and low-rank tensor approximations.

\subsection{Electron repulsion integral tensor}
\label{subsec:eri-tensor}

Let $\{\varphi_\mu\}_{\mu=1}^{N_\mathrm{bas}}$ denote a set of spatial basis
functions (typically Gaussian-type orbitals). The two-electron Coulomb
interaction in this basis is encoded by the electron repulsion integral (ERI)
tensor
\begin{equation}
  (\mu\nu|\lambda\sigma)
  =
  \iint
  \varphi_\mu(\mathbf{r}_1)\varphi_\nu(\mathbf{r}_1)
  \frac{1}{\|\mathbf{r}_1 - \mathbf{r}_2\|}
  \varphi_\lambda(\mathbf{r}_2)\varphi_\sigma(\mathbf{r}_2)
  \, d\mathbf{r}_1\,d\mathbf{r}_2.
\end{equation}
Here $\mu,\nu,\lambda,\sigma \in \{1,\dots,N_\mathrm{bas}\}$ index basis
functions. The ERI tensor is a rank-4 object with
$\Theta(N_\mathrm{bas}^4)$ elements, and it possesses several permutation
symmetries, such as
\begin{equation}
  (\mu\nu|\lambda\sigma)
  = (\nu\mu|\lambda\sigma)
  = (\mu\nu|\sigma\lambda)
  = (\lambda\sigma|\mu\nu).
\end{equation}
By grouping pairs of indices into compound indices, e.g.\ $(\mu\nu)$ and
$(\lambda\sigma)$, one can reshape the ERI tensor into a
$N_\mathrm{bas}^2\times N_\mathrm{bas}^2$ matrix, which is convenient for
numerical linear algebra operations and for applying low-rank and hierarchical
approximations.

The ERI tensor plays a central role in Hartree--Fock, post-HF correlation
methods, and density functional theory. Direct storage of all ERIs scales as
$\Theta(N_\mathrm{bas}^4)$, and the naive computation of all integrals has a
similar or worse cost, depending on the integral algorithm
\cite{Boys1950,HeadGordonPople1988}. This motivates a wide range of integral
screening and compression techniques, including density fitting (resolution of
the identity) \cite{Whitten1973,BeebeLinderberg1977}, Cholesky decomposition of
the ERI matrix \cite{Aquilante2010}, and tensor hypercontraction
\cite{HohensteinMartinez2012}. These methods seek to reduce both the storage
cost and the asymptotic complexity of forming Coulomb and exchange contributions
by exploiting approximate low-rank structure and the locality of basis
functions.

\subsection{Hartree--Fock method}
\label{subsec:hf}

The Hartree--Fock (HF) method provides a mean-field approximation to the
electronic ground state by assuming that the $N$-electron wave function is a
single Slater determinant built from $N$ spin-orbitals. In its spin-restricted,
closed-shell form, HF minimizes the electronic energy with respect to a set of
occupied spatial orbitals $\{\phi_i\}$, subject to orthonormality constraints
\cite{SzaboOstlund1989,HelgakerJorgensenOlsen2000}. This leads to the
Roothaan--Hall equations
\begin{equation}
  \mathbf{F} \mathbf{C} = \mathbf{S} \mathbf{C} \mathbf{\varepsilon},
\end{equation}
where $\mathbf{F}$ is the Fock matrix, $\mathbf{S}$ is the overlap matrix,
$\mathbf{C}$ is the coefficient matrix of molecular orbitals in the atomic
orbital (AO) basis, and $\mathbf{\varepsilon}$ is the diagonal matrix of orbital
energies.

In an AO basis $\{\varphi_\mu\}$, the Fock matrix elements are given by
\begin{equation}
  F_{\mu\nu}
  = h_{\mu\nu}
    + \sum_{\lambda\sigma} P_{\lambda\sigma}
      \left[ 2 (\mu\nu|\lambda\sigma) - (\mu\lambda|\nu\sigma) \right],
\end{equation}
where $h_{\mu\nu}$ are one-electron integrals (kinetic energy and
electron--nuclear attraction), $P_{\lambda\sigma}$ is the density matrix
constructed from occupied orbitals, and $(\mu\nu|\lambda\sigma)$ are electron
repulsion integrals (ERIs). The HF procedure iterates between building the Fock
matrix from a trial density matrix and solving the generalized eigenvalue
problem until self-consistency is reached (self-consistent field, SCF).

The computational bottleneck of conventional HF lies in the construction of the
Coulomb and exchange contributions to the Fock matrix. Without any screening or
compression, the number of significant ERIs scales as
$\Theta(N_\mathrm{bas}^4)$, and each SCF iteration has a formal cost of
$\mathcal{O}(N_\mathrm{bas}^4)$ or higher, where $N_\mathrm{bas}$ is the number
of basis functions \cite{SzaboOstlund1989,HelgakerJorgensenOlsen2000}. Various
techniques have been developed to reduce this cost, including integral
prescreening, density fitting (resolution of the identity)
\cite{Whitten1973,BeebeLinderberg1977}, Cholesky decomposition of the ERI
matrix \cite{Aquilante2010}, and local or linear-scaling HF methods that exploit
the decay of the density matrix with distance and the nearsightedness of
electrons in insulators \cite{Yang1991,Goedecker1999}. Many of these techniques
are compatible with low-rank and hierarchical representations of the ERI tensor,
which further reduce the asymptotic complexity of HF calculations.

\subsubsection{Hierarchical block low-rank and $\mathcal{H}^2$-based ERI representations}
\label{subsubsec:hf-h2-eri}

A more recent line of work by Xing, Huang, and Chow combines the Hartree--Fock
framework with hierarchical block low-rank and $\mathcal{H}^2$-matrix
techniques to obtain near-linear-scaling algorithms for constructing the
Coulomb and exchange matrices.

In their J.~Chem.~Phys.~paper, Xing, Huang, and Chow introduced a
\emph{hierarchical block low-rank representation} of the ERI tensor
\cite{xing2020}. By reshaping the ERI tensor into a matrix with compound
indices $(\mu\nu)$ and $(\lambda\sigma)$, they treat it as a kernel matrix
associated with the Coulomb interaction between products of AO basis functions.
The AO indices are organized into a hierarchical cluster tree, and a
block-cluster tree is used to partition the ERI matrix into near-field and
far-field blocks. Near-field blocks are stored explicitly, while far-field
blocks are approximated in low-rank form using hierarchical block low-rank
techniques closely related to $\mathcal{H}^2$-matrices. This leads to a
data-sparse representation of the ERI tensor in which both storage and
matrix-vector multiplications scale linearly with the matrix dimension (up to
logarithmic factors), rather than quadratically as in the dense case
\cite{xing2020}.

To efficiently construct such $\mathcal{H}^2$-type representations, Huang,
Xing, and Chow developed the H2Pack library for kernel matrices
\cite{huang2020toms}. H2Pack uses a hybrid analytic--algebraic compression
strategy based on the proxy point method to build $\mathcal{H}^2$-matrix
representations with linear complexity in the number of points. Storage and
matrix-vector multiplication costs are both $\mathcal{O}(N)$, where $N$ is the
matrix dimension, under standard assumptions on the kernel and the geometry of
the points \cite{huang2020toms}. Although H2Pack is a general-purpose kernel
matrix package, its underlying ideas can be directly applied to the ERI matrix
viewed as a Coulomb kernel matrix over AO products.

Within the HF context, the hierarchical block low-rank ERI representation
enables the Coulomb and exchange contributions to the Fock matrix to be
constructed using only matrix-vector and small dense-matrix operations with the
compressed ERI representation. As a result, the asymptotic cost of building the
Fock matrix can be reduced from $\Theta(N_\mathrm{bas}^4)$ to nearly linear in
$N_\mathrm{bas}$ for large three-dimensional systems, while controlling the
approximation error through the ranks and tolerance parameters in the
hierarchical compression \cite{xing2020}. From the perspective of this work,
these results illustrate how hierarchical and $\mathcal{H}^2$-based low-rank
representations of the ERI tensor can substantially improve the asymptotic
behavior of mean-field electronic-structure calculations, and they provide an
important point of comparison for the tensor and hierarchical approaches
developed later in this paper.

\subsection{Second-order M{\o}ller--Plesset perturbation theory (MP2)}
\label{subsec:mp2}

Second-order M{\o}ller--Plesset perturbation theory (MP2) is one of the
simplest and most widely used post-HF correlation methods. Starting from a
converged HF reference, MP2 treats the residual electron correlation as a
second-order perturbation \cite{MollerPlesset1934}. In the canonical molecular
orbital (MO) basis, the MP2 correlation energy can be written as
\begin{equation}
  E_\mathrm{MP2}
  = \sum_{ijab}
    \frac{(ij||ab)^2}{\varepsilon_i + \varepsilon_j - \varepsilon_a - \varepsilon_b},
\end{equation}
where $i,j$ index occupied MOs, $a,b$ index virtual MOs, $\varepsilon_p$ are
orbital energies, and $(ij||ab)$ are antisymmetrized two-electron integrals in
the MO basis:
\begin{equation}
  (ij||ab) = (ij|ab) - (ij|ba).
\end{equation}

A straightforward evaluation of $E_\mathrm{MP2}$ scales as
$\mathcal{O}(N_\mathrm{occ}^2 N_\mathrm{vir}^2 N_\mathrm{bas})$ in floating
point operations and requires $\Theta(N_\mathrm{occ} N_\mathrm{vir}
N_\mathrm{bas}^2)$ storage if all relevant ERIs are precomputed, where
$N_\mathrm{occ}$ and $N_\mathrm{vir}$ are the numbers of occupied and virtual
MOs, respectively, and $N_\mathrm{bas}$ is the number of basis functions. In
terms of a single size parameter $N$ representing system or basis size, the
overall scaling is typically described as $\mathcal{O}(N^5)$ in time and
$\mathcal{O}(N^4)$ in memory \cite{SzaboOstlund1989,HelgakerJorgensenOlsen2000}.

Because of this unfavorable asymptotic behavior, a large literature has focused
on reducing the scaling of MP2. Existing approaches include local correlation
methods, which exploit the spatial locality of occupied orbitals and electron
pairs \cite{SaeboPulay1993,SchutzWerner2001}, density fitting and Cholesky
decomposition of the ERI tensor
\cite{Hattig2005,Aquilante2010,Weigend2002}, Laplace-transformed MP2
\cite{Hattig2005,AyalaScuseria1999}, and tensor-factorization techniques such
as resolution-of-the-identity MP2 (RI-MP2) and tensor hypercontraction
\cite{HohensteinMartinez2012}. Many of these methods can achieve effective
scaling closer to $\mathcal{O}(N^4)$ or even lower for large, insulating
systems, while maintaining chemical accuracy.

In this work, MP2 primarily serves as a representative example of a correlated
wave function method whose computational cost is dominated by operations
involving the ERI tensor. The asymptotic behavior of the MP2 energy evaluation
is therefore closely tied to the structure and compression of the ERI tensor
introduced in Section~\ref{subsec:eri-tensor}.